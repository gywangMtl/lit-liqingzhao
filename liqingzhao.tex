\documentclass[14pt]{report}
\usepackage[BoldFont,SlantFont,CJKsetspaces,CJKchecksingle]{xeCJK}
\setCJKmainfont[BoldFont=SimHei]{SimSun}
\setCJKmonofont{SimSun}
\begin{document}
\title{李清照诗词选注}
\maketitle
\tableofcontents
\chapter*{前言}
李清照(1084--约1151),号易安居士,山东济南人。父名格非,字文叔,致力于经学,
以散文见称于时;
\part*{诗}
\chapter{浯溪中兴颂碑诗和张文潜韵二首}
\section{其一}
五十年功如电扫,华清花柳咸阳草。

五坊供奉斗鸡儿,酒肉堆中不知老。

胡兵忽自天上来,逆胡亦是奸雄才。

勤政楼前走胡马,珠翠踏尽香尘埃。

何为出战辄披靡,传置荔枝多马死。

尧功舜德本如天,安用区区纪文字。

著碑铭德真陋哉,乃令神鬼磨山崖。

子仪光弼不自猜,天心悔惑人心开。

夏为殷鉴当深戒,简策汗青今具在。

君不见当时张说最多机,虽生已被姚崇卖。

\section{其二}
君不见,惊人废兴传天宝,中兴碑上今生草。

不知负国有奸雄,但说成功尊国老。

谁令妃子天上来,虢、秦、韩国皆天才。

苑桑羯鼓玉方响,春风不敢生尘埃。

姓名谁复知安史,健儿猛将安眠死。

去天尺五抱瓮峰,峰头凿出开元字。

时移势去真可哀,奸人心丑深如崖。

西蜀万里尚能反,南内一闭何时开。

可怜孝德如天大,反使将军称好在。

呜呼!

奴辈乃不能道:“辅国用事张后专。”
乃能念:“春荠长安作斤卖。”

\chapter{分得知字韵}
学诗三十年,缄口不求知。

谁遣好奇士,相逢说项斯。
\chapter{夏日绝句}
生当作人杰,死亦为鬼雄。

至今思项羽,不肯过江东。
\chapter{咏史}
两汉本继绍,新室如赘疣。

所以嵇中散,至死薄殷周。
\end{document}